\documentclass[10pt,journal]{IEEEtran}

\usepackage{lineno,hyperref}
\modulolinenumbers[5]



\usepackage{graphicx} % for pdf, bitmapped graphics files
\usepackage{epsfig} % for postscript graphics files
%\usepackage{mathptmx} % assumes new font selection scheme installed
%\usepackage{times} % assumes new font selection scheme installed
\usepackage{amsmath} % assumes amsmath package installed
\usepackage{amssymb}  % assumes amsmath package installed
\usepackage{epstopdf}
\usepackage{mathtools}
\usepackage{multirow}
%\usepackage[markup=underlined]{changes}
%\usepackage[final]{changes}
\usepackage{array}
\usepackage{siunitx}
\usepackage{tikz}
\usepackage{mathrsfs}
\usepackage{tabularx}
\usepackage{amsthm}
\usepackage{url}
\usetikzlibrary{calc}
\usetikzlibrary{positioning}
\usetikzlibrary{arrows}
\usetikzlibrary{shapes}
\usetikzlibrary{fit}
\newtheorem{rem}{Remark}
\newtheorem{assm}{Assumption}
\newtheorem{alg}{Algorithm}
\newtheorem{theorem}{Theorem}
\newtheorem{lemma}{Lemma}
\newtheorem{corollary}{Corollary}
\usepackage{algpseudocode}
\usepackage{float}
\usepackage{xcolor}

\usepackage{svg}
\usepackage{psfrag}

\usepackage{pgfplots}
\usetikzlibrary{calc}
\pgfplotsset{compat=1.8}

\definecolor{line1}{cmyk}{0.02,0.16,1,0}
\definecolor{line2}{cmyk}{0.86,1,0.03,0.01}

\graphicspath{{Figures/}}



%%%%%%%%%%%%%%%%%%%%%%%

%\definechangesauthor[color=Red]{RS}
%\definechangesauthor[color=Blue]{CM}

\begin{document}
	
	\title{Robust Game-Theoretic Decision-Making for Autonomous Vehicles}
	
	%% Group authors per affiliation:
	\author{
		Gokul S.~Sankar,  Kyoungseok Han and Ilya Kolmanovsky% <-this % stops a space
		\thanks{Gokul~S.~Sankar,  Kyoungseok Han and Ilya Kolmanovsky are with the Department of Aerospace Engineering, University of Michigan, Ann Arbor, MI, USA. Emails: {\tt\small ggowrisa@umich.edu} (G.~Sankar), 
			{\tt\small kyoungsh@umich.edu} (K.~Han) and  {\tt\small ilya@umich.edu} (I.~Kolmanovsky) }%
	}
	
	
	%============================================================================================%
	
	\maketitle
	\thispagestyle{empty}
	\pagestyle{empty}
	
	
	\begin{abstract}
		
	\end{abstract}
	
	\section{Introduction}
	Despite many recent advances in connected and automated vehicles (CAVs) technology, the full automation systems that can provide the similar or better ability compared to the human drivers are still inherently flawed to be deployed in the market \cite{okuda2014survey}. One of the most significant challenges is to plan the motion of the automated vehicles at the mixed-traffic conditions where the automated car coexists with the human-driven vehicles \cite{lazar2018maximizing}. In particular, describing the human decision-making process is the most difficult since the humans do not always exhibit the optimized behavior due to limited rationality \cite{griffiths2015rational}. In order to ensure the safety, a conservative driving policy of the automated vehicles where the all possible driving situations are considered has been suggested \cite{claussmann2015path, brechtel2014probabilistic}, but it may cause the disharmony with the human drivers. That is, too conservative behavior of the automated vehicles sometimes causes adverse effect on traffics, e.g., road congestion, car accidents, due to the uncertainties of the human drivers.
	
	If the human-driven vehicle's response according to traffics, i.e., behavior of the surrounding vehicles, can be predicted, far less conservative motion planning of the automated vehicle can be designed \cite{sadigh2016planning}. As mentioned, however, modeling of the interaction between the vehicles may not very accurate. Moreover, since the communication time with the surrounding vehicles in reality is not sufficient to build the deterministic human driver models.
	
	To overcome such shortcomings, we exploit the ``level-\textit{k} game theory'' framework where the human thought processes in strategic games, i.e., driving at the mixed-traffics, are modeled in hierarchical structure \cite{stahl1993evolution}. The game theoretic approach has already proved its effectiveness when describing the interactions between the vehicles \cite{li2017game, li2018game, tian2018adaptive}. Although these approaches effectively describe rational decision-making for the human-driven vehicles, the model uncertainties that affect to the vehicle position are not considered. Therefore, it is assumed that the center of gravity of the vehicle follows the deterministic kinematic model with the constant safety constraint that prevents the physical safety violation between the vehicles, and depending on the size of the safety constraint, the level of conservatism in the autonomous vehicle's motion is determined. 
	
	In this paper, we adaptively adjust the safety constraint sizes of the interactive human-driven vehicles according to the confidence level to establish a balanced motion planning between the aggressive and conservative motion planning. The interactive vehicle's aggressiveness level is estimated and it is proportional to the size of safety constraint of the human-driven vehicle. Based on this, the autonomous vehicle's future trajectory is planned, which maximizes the cumulative reward. In \cite{Jinge}, the caution level of the human-driven vehicle is used to describe the aggressiveness of the HDV, but only constant caution level is assumed, which eventually determine the level of conservatism of AV. To resolve this, we incorporate the level-\textit{k} game theory with the caution level so that adaptive caution level is available, which enables adjustable motion planning of AV depending on the interactive vehicle's aggressiveness.
	
	The contributions of this paper are summarized as follows. First and foremost, a balanced motion planning of AV is 
	\textcolor{blue}{(Contributions is not stated yet)}
	
	The rest of the paper is organized as follows. In Section II, the vehicle dynamic model with the bounded uncertainty is introduced. We next present the game-theoretic decision making algorithm and motion planning procedure in Section III, the effectiveness of the proposed approach is described in Section IV, and the paper is concluded in Section V.
	
	\textcolor{red}{red: not super-clear}, \textcolor{blue}{blue: added or modified.}
	\label{sec:intro}
	
	
	
	
	
	
	% 	\subsection{Notation}
	
	\section{Vehicle model \& Problem statement}
	\label{sec:model}
	\subsection{Kinematic bicycle model}
	The  vehicle dynamics are represented by the following discrete kinematic bicycle model \cite{kong2015kinematic}:
	
	\begin{subequations}
		\begin{align}
		x_{t+1} & = x_t + v_t \cos\left(\psi_t + \beta_t \right) \Delta t + \textcolor{blue}{w^x_t}, \\ 
		y_{t+1} & = y_t + v_t \sin\left(\psi_t + \beta_t \right) \Delta t + \textcolor{blue}{w^y_t}, \\ 
		\psi_{t+1} & = 	\psi_t + \frac{v_t}{l_r} \sin\left( \beta_t \right) \Delta t, \\ 
		v_{t+1} & =	v_t + a_t \Delta t, \\
		\beta_t & = \arctan\left(\frac{l_r}{l_r+l_f}\tan\left(\delta_f\right)\right).
		% 	x\left(t+1\right) & = 	x\left(t\right) + v\left(t\right) \cos\left(\psi \left(t\right) + \beta \left(t\right) \right) \Delta t + w_x\left(k\right), \\ 
		% 	y\left(t+1\right) & = 	y\left(t\right) + v\left(t\right) \sin\left(\psi \left(t\right) + \beta \left(t\right) \right) \Delta t + w_y\left(k\right), \\ 
		% 	\psi\left(t+1\right) & = 	\psi\left(t\right) + \frac{v\left(t\right)}{l_r} \sin\left( \beta \left(t\right) \right) \Delta t, \\ 
		% 	v\left(t+1\right) & = 	v\left(t\right) + a\left(t\right)  \Delta t, \\
		% 	\beta\left(t\right) & = \arctan\left(\frac{l_r}{l_r+l_f}\tan\left(\delta_f\left(t\right)\right)\right).
		\end{align}
		\label{eq:mod}
	\end{subequations}
	
	
	\noindent where  $t$ denotes the discrete time instant; the pair $\left(x_t,\,y_t\right)$  represent the global position of the center of mass of the vehicle; the vehicle's speed is denoted by $v_t$; $\beta_t$ is the angle of $v_t$ with respect to the longitudinal axis of the vehicle; $\psi_t$ denotes the vehicle’s yaw angle (the angle between the vehicle’s heading direction and the global x-direction); $a_t$ denotes the vehicle’s acceleration at time $t$;  $\Delta t$ denotes the time step size; $\delta_f$  represents the front steering angle; and  $l_f$  and $l_r$  are the distance of the center of the mass of the vehicle to the front and rear axles, respectively; $w^x_t$  and $w^y_t$ denote the uncertainty in the position of the center of mass, respectively. It is assumed the uncertainties originate from a closed and compact disturbance set, \textcolor{red}{ ${\mathcal{W}} \coloneqq  \left\{ w = \left(w^x,\,w^y\right) |\zeta w\leq\theta,\,\zeta\in\mathbb{R}^{a\times 2},\,\theta\in\mathbb{R}^{b}\right\}$, with $b \in 2\mathbb{Z}^{+}$ }. The disturbance set is assumed to contain the origin. Furthermore, it is assumed that the rear wheels cannot be steered. The control input to the model \eqref{eq:mod} at time step $t$, represented by $\gamma = \left(a ,\, \delta_f \right)$, is the acceleration [$\text{m/s}^2$] and front steering angle [rad] pair. 
	
	\subsection{Problem statement}
	\textcolor{blue}{As illustrated in Fig.~\ref{fig:lane_changing}, depending on the motions of interactive vehicles (red), the autonomous vehicle can plan the different motions. If the interactive vehicles behave aggressively, the AV should choose the conservative motion planning, and vice versa. In addition, to avoid the too conservative or aggressive driving policy, the balanced motion planning is preferred in the actual application.
	}
	
	\textcolor{blue}{The problem we treat is to model the intent of the interactive vehicles using game theoretical approach, then AV exploits the estimated interactive vehicle's driving intention when making the decision in real-time. The confidence level of the estimated intention is represented by dashed-dotted boxes in Fig.~\ref{fig:lane_changing} and their sizes are adjusted adaptively according to the confidence levels.}
	\section{Game-theoretic decision making}
	\label{sec:controller}
	
	At each time instant, each vehicle selects an input pair from the finite action set, \textcolor{blue}{$\boldsymbol{\Gamma}$=\{(0, 0), (0, $\frac{\pi}{225}$), (0, $-\frac{\pi}{225}$), (2.5, 0), (-2.5, 0), (-5, 0), (4, 0), (2.5, $\frac{\pi}{180}$), (2.5, $-\frac{\pi}{180}$)\} } % where $a_{\max}$ and $\delta_{f,\,\max}$ are the maximum acceleration and front steer angle, respectively.% 
	The inputs pairs in $ \boldsymbol{\Gamma}$ correspond to the actions, \{``maintain", ``turn slightly left", ``turn slightly right", ``accelerate", ``decelerate", ``hard decelerate", ``hard accelerate", ``turn left", ``turn right"\}, respectively. The input pair to be applied at every time step is decided based on optimizing a reward function.
	
	
	\subsection{Action choice}
	
	The decision making process of the vehicle in choosing the optimal input pair follows a receding horizon strategy. A sequence of actions, $\boldsymbol{\gamma^{t}} = \left\{\gamma_{t},\,\gamma_{t+1},\,\cdots,\,\gamma_{t+N-1}\right\}$, is chosen that maximizes a cumulative reward given by
	
	\begin{align}
	\mathcal{R}\left(\boldsymbol{\gamma^{t}}\right) = \sum_{j=0}^{N-1} \lambda^{\textcolor{blue}{j}} R_{t+j}\left(\gamma_{t+j}\right),
	\label{eq:cum_reward}
	\end{align}
	
	\noindent where $R_{t+j}\left(\gamma_{t+j}\right)$ is the stage reward at a prediction step $j$ determined at time step $t$ for an input, $\gamma_{t+j} \in  \boldsymbol{\Gamma}$; $\lambda \in \left[0,\,1\right]$ is the discount factor. By the receding horizon strategy, the input applied to \eqref{eq:mod}, $\gamma\left(t\right)$, is the first element of $\boldsymbol{\gamma_{t}}^* = \left\{\gamma_{t}^*,\,\gamma_{t+1}^*,\,\cdots,\,\gamma_{t+N-1}^*\right\}$ is applied at each time instant $t$, i.e.,  $\gamma\left(t\right) = \gamma_{t}^*$. The stage reward at a prediction step $j$, $R_{t+j}\left(\gamma_{t+j}\right)$, is defined as
	
	\begin{align}
	R_{t+j}\left(\gamma_{t+j}\right) = R_{t+j}\left(\gamma_{t+j}| s_{t+j}\right) = \boldsymbol{{\alpha}}^T \boldsymbol{\phi_{t+j}}
	\end{align}
	
	
	
	
	
	\begin{figure}
		\begin{centering}
			\begin{tikzpicture}
			
			\node (rect) [draw =none, very thin, fill = white, minimum width = 8.25cm, minimum height = 3cm, inner sep = 0pt] {};
			
			\node (off_road_north) [draw, very thick, fill = black, minimum width = 8.25cm, minimum height = 0.1mm, inner sep = 0pt, above  = 0mm of rect] {};
			
			\node (off_road_south) [draw, very thick, fill = black!60!green, minimum width = 8.25cm, minimum height = 0.1mm, inner sep = 0pt, below  = 0mm of rect] {};
			
			
			\draw  [loosely dashed, thick] (rect.west) -- (rect.east);
			\node (blue)	[below left = -1.25cm and -2.2cm of rect]{\includegraphics[scale=0.1]{blue_car.pdf}};
			\node (red1) [above left = -1.25cm and -1.8cm of rect] {\includegraphics[scale=0.1]{red_car.pdf}};
			\node (red2)	[above right = -2.75cm and -2cm of rect]{\includegraphics[scale=0.1]{red_car.pdf}};
			
			\node (mid3)	[above left = -1.58cm and -5.9cm of rect, minimum width = 0cm, minimum height = 0cm, inner sep = 0pt] {};
			\node (mid2)	[above left = -1.56cm and -4.9cm of rect, minimum width = 0.cm, minimum height = 0.cm, inner sep = 0pt] {};
			\node (mid1)	[above left = -1.56cm and -3.9cm of rect, minimum width = 0.cm, minimum height = 0.cm, inner sep = 0pt] {};
			
			
			
			
			\node (off_road_north) [draw = none, very thick, fill = black!60!green, minimum width = 8.25cm, minimum height = 1cm, inner sep = 0pt, above  = 0mm of rect, fill opacity=0.6] {};
			
			\node (off_road_south) [draw = none, very thick, fill = black!60!green, minimum width = 8.25cm, minimum height = 1cm, inner sep = 0pt, below  = 0mm of rect, fill opacity=0.6] {};
			
			
			\node (final)	[draw, circle, above right = -0.83cm and -0.5cm of rect, minimum width = 0.cm, minimum height = 0.cm, inner sep = 0pt] {};
			
			\node (final3)	[draw,  above right = -0.83cm and -1.5cm of rect, minimum width = 0.cm, minimum height = 0.cm, inner sep = 0pt] {};
			\node (final2)	[draw, circle, above right = -0.83cm and -2.5cm of rect, minimum width = 0.cm, minimum height = 0.cm, inner sep = 0pt] {};
			\node (final1)	[draw, circle, above right = -0.83cm and -3.5cm of rect, minimum width = 0.cm, minimum height = 0.cm, inner sep = 0pt] {};
			
			
			\node (breakaway1)	[draw, circle,  below left = -0.83cm and -3cm of rect, minimum width = 0.cm, minimum height = 0cm, inner sep = 0pt]{};
			\node (breakaway2)	[draw, circle,  below left = -0.83cm and -4cm of rect, minimum width = 0.cm, minimum height = 0cm, inner sep = 0pt]{};
			\node (breakaway3)	[draw, circle,  below left = -0.83cm and -5cm of rect, minimum width = 0.cm, minimum height = 0cm, inner sep = 0pt]{};		
			
			\draw [rounded corners,color=cyan,line width=3pt] (blue) .. controls (breakaway1) ..  (mid1) .. controls (final1) .. (final2) -- (final);
			\draw [rounded corners,color=line1,line width=3pt] (blue) .. controls (breakaway2) ..  (mid2) .. controls (final2) .. (final3) -- (final);
			\draw [rounded corners,color=line2,line width=3pt] (blue) .. controls (breakaway3) ..  (mid3) .. controls (final3) .. (final) ;
			
			\node[above left = -1.15cm and -2cm of blue,]{A};
			\node[above left = -1.15cm and -2.75cm of blue,]{A1};
			\node[above left = -1.15cm and -3.75cm of blue,]{A2};
			\node[above left = -1.15cm and -4.75cm of blue,]{A3};
			%		
			\node[above left = 1.15cm and -4.5cm of blue,]{B1};
			\node[above left = 1.15cm and -5.5cm of blue,]{B2};
			\node[above left = 1.15cm and -6.5cm of blue,]{B3};
			\node[above left = 1.15cm and -7.5cm of blue,]{B};
			
			\end{tikzpicture}
			\par\end{centering}
		\protect\caption{Possible lane changing sequences; ``aggressive motion planning"= \{(A,A1),\,(A1,B1),\,(B1,B)\}, ``balanced motion planning"=$\{\left(A,A2\right),\,\left(A2,B2\right),\,\left(B2,B\right)\}$ and ``conservative motion planning"=$\left\{\left(A,A3\right),\,\left(A3,B3\right),\,\left(B3,B\right)\right\}$ that can be chosen by the autonomous vehicle (blue) based on the motion of other non-autonomous vehicles (red).}
		\label{fig:lane_changing}
	\end{figure}
	
	
	
	\noindent where $s_{t+j}$ is the traffic state  at prediction step $j$;  $\boldsymbol{{\phi}_{t+j}} = \left\{\phi_{1,\,{t+j}},\,\phi_{2,\,{t+j}},\,\cdots,\,\phi_{m,\,{t+j}}\right\}$ is the feature vector at step $j$ and the weights for these features are in  $\boldsymbol{{\alpha}} = \left\{\alpha_{1},\,\alpha_{2},\,\cdots,\,\alpha_{m}\right\}$, in which $\alpha_{i}>0,\,\forall \, i\in \mathbb{Z}_{\left[0:m\right]}$. 
	
	For the lane changing scenario in  Fig.~\ref{fig:lane_changing}, the  features considered are described below. Rectangular outer approximation of the geometric contour of each vehicle is considered as shown by the dash-dotted boxes in Fig.~\ref{fig:lane_changing}. This outer approximation is referred as the collision avoidance zone (c-zone). The features \textcolor{blue}{when \textit{m}=6, i.e.,} $\phi_{1,\,{t}},\,\phi_{2,\,{t}}$ and $\phi_{3,\,{t}}$, are indicator functions based on the c-zone of the vehicles that respectively characterize:
	\begin{itemize}
		\item Collision status - The intersection of the c-zone of the ego vehicle with that of any other vehicle indicates a collision or a danger of collision. If an overlap is detected then $\phi_{1,\,{t}}$ is assigned a value $-1$; and $0$ otherwise.
		
		\item On-road status - The intersection of the c-zone of the ego vehicle with that of green regions shown in Fig.~\ref{fig:lane_changing} indicates that the ego vehicle is outside the road boundaries. The feature $\phi_{2,\,{t}} = -1$ if an overlap is detected; $\phi_{2,\,{t}} = 0$, otherwise.
		
		\item Safe zone violation status - A safe zone (s-zone) of a vehicle is a rectangular area that subsumes the c-zone of the vehicles with a safety margin. The safety margin is chosen based on the minimum distance to be maintained from the surrounding vehicles. If an overlap of the s-zone of the ego vehicle with that of another vehicle is detected then $\phi_{3,\,{t}}$ is assigned a value $-1$; and $0$, otherwise.		
		
		
	\end{itemize}
	
	The other features considered in this work characterize:
	\begin{itemize}
		\item Distance to objective - In order to encourage the ego vehicle to change lane and reach a reference point in the new lane, $\left(x^{\textrm{ref}},\,y^{\textrm{ref}}\right)$, the feature $\phi_{4,\,{t}} $ is defined as
		\begin{align}
		\phi_{4,\,{t}}  = -\left(\left|x_{t}-x^{\textrm{ref}}\right| + \left|y_{t}-y^{\textrm{ref}}\right|\right).
		\end{align}
		
		\item Distance to lane center - The feature, $\phi_{5,\,{t}} $, defined as 
		\begin{align}
		\phi_{5,\,{t}}  = - \left|y_{t}-y^{lc}\right|,
		\end{align}
		
		\noindent where $y^{lc}$ is the y-coordinate of the center of the current lane that is included to encourage the ego vehicle to be at the middle of the current lane.
		
		\item Velocity error - The deviation of the velocity of the ego vehicle from a reference velocity, $v^{\textrm{ref}}$, is described by the feature $\phi_{6,\,{t}}$ as
		\begin{align}
		\phi_{6,\,{t}} = - \left|v_{t}-v^{\textrm{ref}}\right|,
		\end{align}
		where the reference velocity is typically chosen as the legislated speed limit.
	\end{itemize}
	
	\textcolor{blue}{(Up to here, Kyoungseok)}
	\subsection{Level-k framework}
	\label{sec:level_k}
	
	
	
	
	The state of the traffic, $s_{t+j}$, at prediction steps $j=0,\,1,\,\cdots,\,N-1$ have to be known to compute the cumulative reward in \eqref{eq:cum_reward}.
	
	
	, where 
	
	\begin{align}
	\boldsymbol{\gamma_{t}}^* = \arg \underset {\boldsymbol{\gamma_{t}} \in  \boldsymbol{\Gamma}} {\max}\, \mathcal{R}\left(\boldsymbol{\gamma_{t}}\right),
	\end{align}
	
	\noindent
	
	
	We use a game theoretic model for this prediction.
	
	Numerous experimental results from psychology, cognitive science, and economics have suggested a hierarchical struc- ture in human reasoning in games, see [13], [14], [15], [16]. The study of this reasoning hierarchy and its applications in game theoretic settings are addressed by the “level-k game theory.” In [5], [6], [17], the level-k game theory is exploited to model vehicle interactions in highway traffic. The model has been compared to human traffic data in [6]. In this paper, we also exploit level-k game theory to model vehicle interactions, in particular, at intersections. Recently, level-k modeling of human agents was also considered in aerospace and energy applications [18], [19], [20], where human-to- human and human-to-automation interactions play a central role.
	
	
	The model is premised on the idea that strategic agents (drivers/vehicles at an intersection, in our setting) have different reasoning levels. In particular, the level k indicates an agent’s reasoning depth. The reasoning hierarchy starts from level-0. A level-0 agent makes instinctive decisions to pursue its goal without considering the interactions between itself and the others. On the contrary, a level-1 agent takes into account such interactions in its decision making process, in particular, by assuming that all the other agents in the game are level-0. Specifically, a level-1 agent assumes that all the other agents are level-0 so they make instinctive decisions; the level-1 agent predicts their actions as well as the evolutions of the game resulting from their actions based on this assumption; the level-1 agent then makes its own decision as the best response to such evolutions to pursue its own goal. Similarly, a level-k agent assumes that all the other agents are level-(k-1), makes predictions based on this assumption, and responds accordingly.
	
	% 	In this paper, a level-0 driver/vehicle treats the other vehicles at the intersection as stationary obstacles, and selects its action sequence, �0, accordingly. In this setting, a level- 0 driver/vehicle may represent an aggressive driver in real traffic, who usually assumes that other drivers will yield the right of way.
	
	% 	After the level-0 is defined, the action selection procedure for level-k, k ≥ 1, in the case of 2-agent interactions has the following form,
	% 	
	% 	
	% 	Based on the definitions of the features, the rewards of a vehicle not only depend on its own states and actions, but also depend on the states and actions of its opponent vehicle (e.g., φ1(t) and φ4(t)). Such an interdependence reflects the interactive nature of vehicle decision making in a multi- vehicle traffic scenario. The receding-horizon optimal control problem is thus formulated as: At each time step t, to select
	% 	
	% 	
	% 	represents the state of the traffic, which contains both the ego vehicle’s states and the opponent vehicle’s states; γego(t + j) is the ego vehicle’s action at t + j over the horizon and is to be optimized, and γopp.(t + j) is the opponent vehicle’s action at t + j over the horizon and is to be predicted. We note that for the two interacting vehicles, either is the “ego vehicle” from its own perspective, and is also the “opponent vehicle” from the other’s perspective, that is, (4) can be used to describe the decision making of either of the two vehicles.
	% 	
	%    
	
	
	\begin{align}
	u_{l}^K = \arg \, & \underset{u_{l} \in \mathcal{U}}{\max} \sum _{j=0}^{N-1} \gamma ^j R\left(x_{l}\left(j+1\right), x_{i\neq l}\left(j+1\right), u_{l}\left(j\right), u_{i \neq l}^{K-1}\left(j\right) \right)\nonumber
	\end{align}
	
	
	\subsection{Driver model identification}
	
	\begin{align}
	&\tilde{K}\left(t\right) = \arg \, \underset{K \in \left\{0,\,1 \right\}} {\min} \left\| u_{i}^{\text{actual}} \left(t\right) - u_{i}^{K} \left(t\right) \right\| \nonumber \\ 
	& P_{K_i = k}\left(t\right) = P_{K_i = k}\left(t-1\right) + I_{\tilde{K}_i\left(t\right) = k} \Delta P \nonumber \\ 
	& P_{K_i = k}\left(t\right) = \frac{P_{K_i = k}\left(t\right) }{\sum_{k' = 0}^{1} P_{K_i = k'}\left(t\right) }\nonumber 
	\end{align}
	
	
	Modified control policy taking model identification into account
	
	\begin{align}
	{u}_{l}^D = \arg \, & \underset{u_{l} \in \mathcal{U}}{\max} \sum_{k = 0}^{1} P_{K_i = k}\left(t\right) \left[ \sum _{j=0}^{N-1} \gamma ^j R\left(x_{l}\left(j+1\right), x_{i\neq l}\left(j+1\right), u_{l}\left(j\right), u_{i \neq l}^{K_i}\left(j\right) \right)\right]\nonumber
	\end{align}
	
	
	
	
	
	\subsection{Robust approach}
	
	
	\begin{align}
	\mathcal{W}_i' = P_{K_i = 0} \cdot \mathcal{W}_i \nonumber 
	\end{align}
	
	
	\begin{align}
	\tilde{u}_{l}^D = \arg \, & \underset{u_{l} \in \mathcal{U}}{\max} \,\,\underset{w_x,\, w_y \in \mathcal{W}'_i}{\min} \left[ \sum_{k = 0}^{1} P_{K_i = k}\left(t\right) \left[ \sum _{j=0}^{N-1} \gamma ^j R\left(x_{l}\left(j+1\right), x_{i\neq l}\left(j+1\right), u_{l}\left(j\right), u_{i \neq l}^{K_i}\left(j\right), w_x,\,w_y \right)\right]\right]\nonumber
	\end{align}
	
	
	
	\section{Simulation results}
	\label{sec:sim_results}
	The effectiveness of the proposed approach is confirmed through the case studies. Starting from the same initial conditions, 
	\begin{figure*}
		\begin{centering}
			\begin{tikzpicture}[scale=0.4,transform shape]
			\node (origin) at (0,0) {};
			
			\node (adp1)[below left = 0cm and 0cm of origin, minimum width = 0.cm, minimum height = 0cm, inner sep = 0pt]{\includegraphics[clip, trim = {1.5cm 0.25cm 1.5cm 0.25cm}]{plot_adp0.jpg}};
			\node (agg1)[left = 1cm of adp1, minimum width = 0.cm, minimum height = 0cm, inner sep = 0pt]{\includegraphics[clip, trim = {1.5cm 0.25cm 1.5cm 0.25cm}]{plot_agg0.jpg}};
			\node (con1)[ right = 1cm of adp1, minimum width = 0.cm, minimum height = 0cm, inner sep = 0pt]{\includegraphics[clip, trim = {1.5cm 0.25cm 1.5cm 0.25cm}]{plot_con0.jpg}};
			
			
			\node (adp2)[below  = 1.cm of adp1, minimum width = 0.cm, minimum height = 0cm, inner sep = 0pt]{\includegraphics[clip, trim = {1.5cm 0.25cm 1.5cm 0.25cm}]{plot_adp2.jpg}};
			\node (agg2)[left = 1cm of adp2, minimum width = 0.cm, minimum height = 0cm, inner sep = 0pt]{\includegraphics[clip, trim = {1.5cm 0.25cm 1.5cm 0.25cm}]{plot_agg2.jpg}};
			\node (con2)[ right = 1cm of adp2, minimum width = 0.cm, minimum height = 0cm, inner sep = 0pt]{\includegraphics[clip, trim = {1.5cm 0.25cm 1.5cm 0.25cm}]{plot_con2.jpg}};
			
			
			\node (adp3)[below  = 1.cm of adp2, minimum width = 0.cm, minimum height = 0cm, inner sep = 0pt]{\includegraphics[clip, trim = {1.5cm 0.25cm 1.5cm 0.25cm}]{plot_adp4.jpg}};
			\node (agg3)[left = 1cm of adp3, minimum width = 0.cm, minimum height = 0cm, inner sep = 0pt]{\includegraphics[clip, trim = {1.5cm 0.25cm 1.5cm 0.25cm}]{plot_agg4.jpg}};
			\node (con3)[ right = 1cm of adp3, minimum width = 0.cm, minimum height = 0cm, inner sep = 0pt]{\includegraphics[clip, trim = {1.5cm 0.25cm 1.5cm 0.25cm}]{plot_con4.jpg}};
			
			
			\node (adp4)[below  = 1.cm of adp3, minimum width = 0.cm, minimum height = 0cm, inner sep = 0pt]{\includegraphics[clip, trim = {1.5cm 0.25cm 1.5cm 0.25cm}]{plot_adp6.jpg}};
			\node (agg4)[left = 1cm of adp4, minimum width = 0.cm, minimum height = 0cm, inner sep = 0pt]{\includegraphics[clip, trim = {1.5cm 0.25cm 1.5cm 0.25cm}]{plot_agg6.jpg}};
			\node (con4)[ right = 1cm of adp4, minimum width = 0.cm, minimum height = 0cm, inner sep = 0pt]{\includegraphics[clip, trim = {1.5cm 0.25cm 1.5cm 0.25cm}]{plot_con6.jpg}};
			
			\node (adp5)[below  = 1.cm of adp4, minimum width = 0.cm, minimum height = 0cm, inner sep = 0pt]{\includegraphics[clip, trim = {1.5cm 0.25cm 1.5cm 0.25cm}]{plot_adp8.jpg}};
			\node (agg5)[left = 1cm of adp5, minimum width = 0.cm, minimum height = 0cm, inner sep = 0pt]{\includegraphics[clip, trim = {1.5cm 0.25cm 1.5cm 0.25cm}]{plot_agg8.jpg}};
			\node (con5)[ right = 1cm of adp5, minimum width = 0.cm, minimum height = 0cm, inner sep = 0pt]{\includegraphics[clip, trim = {1.5cm 0.25cm 1.5cm 0.25cm}]{plot_con8.jpg}};
			%			
			%			\node (adp6)[below  = 1.5cm of adp5, minimum width = 0.cm, minimum height = 0cm, inner sep = 0pt]{\includegraphics[clip, trim = {1.5cm 0.25cm 1.5cm 0.25cm}]{plot_adp10.jpg}};
			%			\node (agg6)[left = 1cm of adp6, minimum width = 0.cm, minimum height = 0cm, inner sep = 0pt]{\includegraphics[clip, trim = {1.5cm 0.25cm 1.5cm 0.25cm}]{plot_agg10.jpg}};
			%			\node (con6)[ right = 1cm of adp6, minimum width = 0.cm, minimum height = 0cm, inner sep = 0pt]{\includegraphics[clip, trim = {1.5cm 0.25cm 1.5cm 0.25cm}]{plot_con10.jpg}};
			
			
			\node (xlabeladp) [below = 1cm of adp5, inner sep = 0pt] {\Huge Distance (m)};
			\node (xlabelagg) [below = 1cm of agg5, inner sep = 0pt] {\Huge Distance (m)};
			\node (xlabelcon) [below = 1cm of con5, inner sep = 0pt] {\Huge Distance (m)};
			
			
			\node (t0) [left = 0.25cm of agg1, inner sep = 0pt] {\Huge $t=0$};
			\node (t1) [left = 0.25cm of agg2, inner sep = 0pt] {\Huge $t=1$};
			\node (t2) [left = 0.25cm of agg3, inner sep = 0pt] {\Huge $t=2$};
			\node (t3) [left = 0.25cm of agg4, inner sep = 0pt] {\Huge $t=3$};
			\node (t4) [left = 0.25cm of agg5, inner sep = 0pt] {\Huge $t=4$};
			
			\node (agg) [above = 0.25cm of agg1, inner sep = 0pt] {\Huge (a) Aggressive};
			\node (adp) [above = 0.25cm of adp1, inner sep = 0pt] {\Huge (b) Adaptive};
			\node (con) [above = 0.25cm of con1, inner sep = 0pt] {\Huge (c) Conservative};
			
			
			\end{tikzpicture}
			\par\end{centering}
		\protect\caption{.}
		\label{snapshots}
	\end{figure*}
	
	
	
	
	\section{Conclusions}
	\label{sec:conclusions}
	
	\section*{Acknowledgement}
	We would like to express out appreciation to the Nan Li who provided expertise that greatly assisted the research and it should be noted that the code publicly available (\url{https://github.com/gokulsivasankar/RobustDecisionMaking}) is based on his previous study \cite{li2018game}.
	\bibliographystyle{ieeetr}
	\bibliography{Reference}
	
	%	\noindent where  $t$ denotes the discrete time instant; the pair $\left(x\left(t\right),\,y\left(t\right)\right)$ $\left[\SI{}{\meter}\right]$ represent the global position of the center of mass of the vehicle; the vehicle's speed is denoted by $v\left(t\right)$ $\left[\SI{}{\meter \per \second}\right]$; 	$\beta\left(t\right)$ $\left[\SI{}{\radian}\right]$ is the angle of $v\left(t\right)$ with respect to the longitudinal axis of the vehicle; $\psi\left(t\right)$ $\left[\SI{}{\radian}\right]$ denotes the vehicle’s yaw angle (the angle between the vehicle’s heading direction and the global x-direction); $a\left(t\right)$ $\left[\SI{}{\meter \per \second^2}\right]$ denotes the vehicle’s acceleration at time $t$;  $\Delta t$ $\left[\SI{}{\second}\right]$ denotes the time step size; $\delta_f\left(t\right)$ $\left[\SI{}{\radian}\right]$ represents the front steering angle; and  $l_f$ $\left[\SI{}{\meter}\right]$ and $l_r$  $\left[\SI{}{\meter}\right]$ are the distance of the center of the mass of the vehicle to the front and rear axles, respectively; $w_x\left(k\right)$ $\left[\SI{}{\meter}\right]$  and $w_x\left(k\right)$ $\left[\SI{}{\meter}\right]$ denote the uncertainty in the position of the center of mass, respectively. It is assumed the uncertainties originate from a closed and compact disturbance set, ${\mathcal{W}} \coloneqq  \left\{ w = \left(w_x,\,w_y\right) |\zeta w\leq\theta,\,\zeta\in\mathbb{R}^{a\times 2},\,\theta\in\mathbb{R}^{b}\right\}$, with $b \in 2\mathbb{Z}^{+}$. The disturbance set is assumed to contain the origin. Furthermore, it is assumed that the rear wheels cannot be steered. The acceleration, $a\left(t\right)$, and the front steering angle, $\delta_f\left(t\right)$, are the two inputs. At a given time instant, the vehicle selects the inputs from an action set as described below.
	
\end{document}